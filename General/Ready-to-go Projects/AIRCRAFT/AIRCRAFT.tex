\documentclass{article}
\usepackage{nips06submit}
%\usepackage[utf8]{inputenc}
\usepackage[catalan]{babel}
\usepackage{url}
\usepackage{verbatim}
\usepackage{hyperref}

\title{COMPUTATIONAL INTELLIGENCE (CI-MAI) - 2016-2017\\
Project proposal: Fuzzy Control System Design for Aircraft Landing}
\author{\`Angela Nebot, Llu\'{\i}s A. Belanche}
%{\small Master in Artificial Intelligence\\
%Barcelona School of Computer Science\\ %(\emph{Facultat d'Inform\`atica})

\begin{document}

\maketitle


\begin{abstract}
  The goal of this project is to develop a simple automatic aircraft
  landing system, in which the aircraft follows a desired landing
  profile. To this end, you should design and develop a fuzzy control
  system that allows an aircraft descend from altitude promptly
  in a very soft manner, taking as state variables the height and the
  vertical velocity of the aircraft. In order to perform the whole
  system simulation, it is necessary that you model the behaviour of
  the aircraft in a very simple way, following the movement equations
  described in this document. The complete simulation should be done
  under \texttt{Simulink}.
\end{abstract}

\section{Introduction}

Most control situations are more complex than we can deal with from
the mathematical modelling point of view. In this situation, a fuzzy
control system can be developed, provided a body of knowledge about the
control process exists, and it can be expressed into a number of fuzzy
rules. Control applications are arguably the kinds of problems for which fuzzy
logic has had the greatest success and acclaim. Many of the consumer
products that we use today involve fuzzy control.

In this project you will conduct a simulation of the final descent and
landing approach of an aircraft. The desired downward velocity is
proportional to the square of the height. Thus, at higher altitudes, a
large downward velocity is desired. As the height (altitude)
decreases, the desired downward velocity should get smaller and smaller. In
the limit, as the height becomes vanishingly small, the downward
velocity also goes to zero. In this way, the aircraft will descend
from altitude promptly but will touch down very gently to avoid
damage.

The two state variables for this simulation will be the height above
ground, {\em h}, and the vertical velocity of the aircraft, {\em
  v}. The control output will be a force that, when applied to the
aircraft, will alter its height, {\em h}, and velocity, {\em v}. The
differential control equations are loosely derived as follows. Mass
{\em m} moving with velocity {\em v} has momentum {\em p = mv}. If no
external forces are applied, the mass will continue in the same
direction at the same velocity, {\em v}. If a force {\em f} is applied
over a time interval $\Delta t$, a change in velocity of
$ \Delta v = f \Delta t/m$ will result. If we let $\Delta t = 1.0$ (s)
and $m = 1.0 $ (lb s$^2$ ft$^{-1}$), we obtain $\Delta v = f$ (lb), or
the change in velocity is proportional to the applied force.

\noindent
In difference notation, we get:

$$v_{i+1} = v_i + f_i$$
$$h_{i+1} = h_i + v_i \cdot \Delta t$$

where $v_{i+1}$ is the new velocity, $v_i$ is the old velocity,
$h_{i+1}$ is the new height, and $h_i$ is the old height. These two
“control equations” define the new value of the state variables {\em
  v} and {\em h} in response to control input and the previous state
variable values.

\section{Assumptions in a Fuzzy Control System Design}

A number of assumptions are implicit in a fuzzy control system design. Five basic assumptions
are commonly made whenever a fuzzy rule-based control policy is selected:

\begin{enumerate}
\item The plant is observable and controllable: state, input, and output variables are usually
available for observation and measurement or computation.
\item There exists a body of knowledge comprising a set of linguistic rules, engineering
common sense, intuition, or a set of input--output measurements data from which
rules can be extracted.
\item A solution exists.
\item The control engineer is looking for a “good enough” solution, not necessarily the
optimum one.
\item The controller will be designed within an acceptable range of precision.

\end{enumerate}


\section{Development of the Fuzzy Logic Controller System}

You are asked to design a simple fuzzy controller and to implement, using \texttt{Simulink}, the close loop system. You can use the 
basic aircraft behaviour equations decsribed above to define your plant. The main steps in designing a simple fuzzy control system are as follows:

\begin{enumerate}
\item Identify the variables (inputs, states, and outputs) of the plant.
\item  Partition the universe of discourse or the interval spanned by each variable into a
number of fuzzy subsets, assigning each a linguistic label (subsets include all the
elements in the universe).
\item Assign or determine a membership function for each fuzzy subset.
\item Assign the fuzzy relationships between the inputs’ or states’ fuzzy subsets on the one
hand and the outputs’ fuzzy subsets on the other hand, thus forming the rule-base.
\item Choose appropriate scaling factors for the input and output variables in order to normalize
the variables to the $[0, 1]$ or the $[-1, 1]$ interval.
\item Fuzzify the inputs to the controller.
\item Use fuzzy approximate reasoning to infer the output contributed from each rule.
\item Aggregate the fuzzy outputs recommended by each rule.
\item Apply defuzzification to form a crisp output.
\end{enumerate}

\end{document}
